\documentclass[twocolumn,prl,nobalancelastpage,aps,10pt]{revtex4-1}
%\documentclass[rmp,preprint]{revtex4-1}
\usepackage{graphicx,bm,times}

\begin{document}

\title{Damage mitigation/elimination and infrastructure recovery in lunar elevator counterweight severance scenario}

\author{A. W. Dawn MSci TPS CIHT, N. J. Isber BS BA}

\affiliation{Copernicus I, Liftport}

\begin{abstract} An essential design consideration in the construction of a nearside lunar elevator is in the positioning of the counterweight used to maintain tension in the tether. Position is dependent on tidal forces, tension requirements, limits on ultimate tensile strength of tether and the available mass for the counter weight. This paper details the additional restriction in altitude of the counter weight to ensure than in the event of the tether breaking that the resultant orbit the counter weight will enter will have perapsis at an altitude above that of the earth's atmosphere and common satellite orbits. This restriction will require a increase in mass of the counter-weight the alternative system explored is a active system of orbital manoeuvres that will allow the counter weight to be placed in an orbit that would otherwise impact Earth.

\end{abstract}
\date{\today}

\maketitle

\section{INTRODUCTION}

The concept of the space elevator is a concept that has been around for more than 100 years, and over the last two decades there have been reasonable amount of research into the physics, materials properties, construction method and operation of such an mega-project \cite{Aravind}. The biggest hurdle yet to be overcome is the development of a material strong and light enough to build a tower or tether with a feasible amount of material. The attraction of the lunar elevator is that the reduced gravity means that construction is feasible with existing materials.

While similar in concept the two elevators require different sources of tension, an earth elevator has a high enough rotational period that centripetal force of a counter weight can be used to produce the tension. As the moon rotates much slower lunar elevator designs use the gravitational force of the earth pulling on the counter weight to produce the tension.

\begin{figure}
\includegraphics*[width=0.96\linewidth,clip]{LunarElavatorSchematic}
\caption{Schematic of lunar elavator design} \label{LunarElavatorSchematic}
\end{figure}

\section{KINETIC IMPACT}
The issue of large mass objects impacting earth has been much explored the gravitation well produces a lower limit on impact velocity. "Rods from God" and meteorites are two examples of extremely serious kinetic impacts. Initial calculations of the impact put the impact at $~ 10 kT$ of TNT, while this is a large amount of energy it only ranks as 0 on the Torino Scale the entire energy will entirely dissipated in the atmosphere with the counter weight completely burning up. For comparison  the 2013 Chelyabinsk meteor had an impact energy of $500kT$ [citation needed].

While the counter weight possesses no threat to people or property on the ground, the loss of a component of the space elevator with that mass would require a very costly replacement.

A design that prevent such a costly situation is a mission critical with two initial options that will eliminate risk, an active or a passive system. A passive system has immediate advantages than an elevator can be designed so that severed parts will not enter a collision orbit and is hence not reliant on an active system that could suffer from component failure.

Worst case scenario is a counter weight just on the nearside of L1 starting from stationary, the potential energy is given as, a real case will never be this bad as at L1 the counter weight will have sufficient velocity to remain orbital:
\begin{equation}
E.P = - \frac{GMm}{r}
\end{equation}
as the object falls this potential is converted into kinetic energy given the equation:
\begin{equation}
\Delta E = \frac{GMm}{r_i} - \frac{GMm}{r_e}
\end{equation}
where $r_e$ is the radius of the earth and $r_i$ is the initial position. Solving this equation gives $14 kTon$ of TNT per ton of counter weight. Not only is the impact catastrophic the calculation is also true in reverse that the same energy is required to lift the counter-weight into position, this is a significant amount of energy and is likely to be the single highest mission cost. A mission undertaking is unlikely to have the budget to build and launch a replacement counter-weight, hence recovery and repair is essential to prevent mission failure.

\section{DAMAGE MITIGATION OF ELEVATOR}

There are three objects along to tether: the counter weight, L1 Station, and the climber. The scenario is dependent on the position of the elevator and the location of the severance. The most likely point of tether failure is at low altitudes where micrometeorite impact and thermal fatigue due to the extreme temperature difference in the lunar day/night cycle. There different scenarios are presented in the table below:

\begin{table}[h!]
  \begin{center}
    \caption{Severance Scenarios}
    \label{tab:scenarios}
    \begin{tabular}{|p{0.25\linewidth}|p{0.25\linewidth}|p{0.25\linewidth}|}
      \hline
      Situation & Climber Low & Cliber High\\
      \hline
      Severed Above L1 & Jettison below L1, climber detaches and initiates powered landing or orbital insertion as suited. & Sever tether below climber, stabilise climber and L1 system.\\
      \hline
      Severed Below L1 & Jettison part/all of counter weight, stabilise L1 and initiates powered landing or orbital insertion as suited. & Jettison part/all of counter weight, stabilise climber and L1 system \\
      \hline
    \end{tabular}
  \end{center}
\end{table}

Severance below the L1 station will require part or all of the counterweight to be detached for the remaining system to have a centre of mass at L1. If the elevator climber is below the sever point to avoid being "dragged down" with the tether it will detach and enter a lunar orbit or require a powered landing (an essential safety feature for a human rated climber) depending on its altitude. Below the height required for stable orbit will also have the option of powered landing or orbital burn to enter an orbit.

Severance above the L1 station will either constant require thrust to maintain tension or jettison of the lower cable and/or climber.

All three objects are recoverable.Small sections of tether will also remain but replacement will be better than repair to avoid weak points. As the tether is purely material the construction cost is marginal compared to the complexity in the objects which are each effectively a spacecraft. Of these two have zero risk to the earth and earth orbiting satellites hence this paper will concentrate on the mitigation options for the counter-weight.

\section{ORBITAL MECHANICS OF PASSIVE DESIGN}

Using Newtonian mechanics it is possible from the initial conditions required by a lunar counter weight to calculate the orbit the weight would enter. This orbit is a function of counter-weight altitude which becomes the apoapsis, and the initial velocity equal to $r\dot{\theta}$ where $\dot{\theta}$. From this condition it is possible calculate the closest approach to the earth periapsis.

This calculation was verified in a numerical simulation using NASA's GMAT to account for perturbation due to the moon using a 3 body system. The effect of the Moon aids the passive design and will narrowly avoid the earth where an analytical model would impact the earth.

In addition to avoiding the earth it is also important to avoid near earth satellites, considering geostationary orbits in the worst case scenario where the orbit of the tether has the same inclination as geostationary orbit so it passes twice through the ring, the probability of getting within $x$ metres of a geostationary satellite is:
\begin{equation}
prob = 2 \frac{nx}{2\pi r}
\end{equation}

where $n$ is the number of satellites, putting in values for the current number of satellites which is $403$ gives the 1 in 1,000 probability of passing within 1km and 1 in 10,000 of within 100m. Note this calculation is assuming that the periapis is sufficiently within the geostationary ring that the counter weight will pass both times, instantly. Also note that these calculations are for the worst case and that the orbit will have an inclination so that the orbit will pass above or below the geostationary ring hence having a much smaller likelihood of impact. This calculation is similar to considering the probability of a Near Earth Object NEO colliding with a satellite.

As active satellites have control over their orbit even in the event of a collision course, evasive action will be possible by the satellite however this manoeuvre has a cost in fuel and will shorten the operational life of the satellite. The operator who has to avoid the counter weight will be able to charge the owner of the counter weight for the cost in the shortening of mission life [citation needed]

\section{MISSION PROFILE OF ACTIVE DESIGN}

Below the safe threshold height the counter weight would have to burn progade at the moment of severance avoid collision with the earth, a second burn at closest approach will allow to return the counter weight to L1 where a final burn will allow the weight to remain there until the tether can be repaired. The $\Delta V$ requirements are dependent on the position of the counter weight. These three manoeuvres are all Hohmann transfer orbits and are considered first for their simplicity and fuel efficiency the later an essential requirement due to the large mass of the counter weight. A continuous burn with a low thrust and high specific impulse engine may more fuel efficient.

Spiting the manoeuvre takes advantage of the Orbeth effect in that manoeuvres are fuel efficient closer to the earth, and is more fuel efficient than immediately circularising the counter weight's orbit.

One design saving that is possible is having part of the tether mass as fuel, however this would then require the repair at L1 to involve refuelling.

\section{SIMULATION DETAILS}

\section{RESULTS}

\section{DISCUSSION}

One of the deciding factors is the difference in mass requirements of an active and passive system...

%todo add comparison of models

however reliability is of essential importance so (despite/additionally) to mass requirements a passive solution is the best option. This has the added benefit of a shorter tether.

Fuel and replacement material costs will be strongly dependent on where the resources can be sourced from. If the elevator had been operational long enough for the L1 station to accumulate fuel process from lunar volatiles the active system would be far cheaper than if fuel had to be sourced from Earth.

\section{CONCLUSIONS}

While tether severance is a legitimate concern, the lunar elevator can be designed so that the counter weight is recoverable. A passive design is attractive as there is not possibility of component failure resulting in burn up in earth's atmosphere. The exact position of the tether should also be positioned so that the orbital period of the severed tether and the lunar period is an integer ratio so that the tether is on a free return to its original location where it can be repaired.

\section{REFERENCES}

\bibliography{LunarElavatorCounterWeight}

\end{document}
